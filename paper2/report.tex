\documentclass[sigconf]{acmart}

\input{format/i523}

\graphicspath{ {images/} }

\usepackage{listings}

\lstset{frame=tb,
  aboveskip=3mm,
  belowskip=3mm,
  showstringspaces=false,
  columns=flexible,
  basicstyle={\small\ttfamily},
  breaklines=true,
  breakatwhitespace=true,
  tabsize=3
}

\begin{document}

\title{Mini Project: ESP8266 And Raspberry Pi Robot Car}

\author{Mani Kumar Kagita}
\affiliation{%
  \institution{Indiana University}
  \streetaddress{107 S. Indiana Avenue}
  \city{Bloomington} 
  \state{Indiana} 
  \postcode{43017-6221}
}
\email{mkagita@iu.edu}


\begin{abstract}
Robotics is the fastest and emerging field in modern world. By providing sufficient intelligence to an autonomous driving robot, it must be able to detect the obstacles coming in its path and should avoid them. Ultrasonic distance sensors provide data to autonomously detect obstacles and avoid in an unstructured environment without having human interference or guidance. A 3 wheeled, two gear motors are used to build the Autonomous Intelligent Robot which is controlled through WIFI using ESP8266 controller and Raspberry Pi. An Arduino program is constructed in ESP8266 NodeMCU board that can produce a multi-angle movement for the Robot to freely move in all directions and also choosing directions to turn based on the detected obstacles. 

\end{abstract}

\keywords{Raspberry Pi, Robot Car, ESP8266 NodeMCU, I523, HID319}

\maketitle

\section{Introduction}
A Robot car is a electro-mechanical machine which is usually guided by a computer program and electronic devices. The main objective of any autonomous robot car is to be navigated in any structured or unstructured environments by avoiding obstacles and prevent collisions. 
Collision avoidance is the process in preventing a vehicle from colliding with any other vehicle or object. 
To be effective, the distance from robot car to the obstacle should be constantly measured in real-time as position of autonomous robot car changes with the time.
Proximity ultrasonic distance measuring sensors are integrated to robot car so as to determine the distance to an object. Using ESP8266 NodeMCU module and webserver to control robot car over wifi. A computer program will be written into this module using wifi too instead of connecting USB cable to laptop. The main advantage of robot car is to stream live videos to the user over wifi and also for security purposes when added a camera to it.


\section{COLLISION AVOIDANCE}
Collision avoidance is said to be one of the driving factors for the design of robot cars. A collision avoidance systems is designed to reduce the collision severity which is also know as pre-crash system or forward collision warning system. Proximity sensors will send data to the robot car on the distance to obstacle. Once the obstacle is detected, robot  car will autonomously take action by halting its movement and check for the direction which is feasible for it to move. 

\section{System Architecture}
System Architecture consists of following blocks :
\begin{description}
    \item[$\bullet$] Raspberry Pi

    \item[$\bullet$] ESP8266 NodeMCU board

    \item[$\bullet$] 3 wheel Robot Car kit

    \item[$\bullet$] L298N DC Stepper Motor Drive Controller 

    \item[$\bullet$] 12v and 5v DC batteries
    
    \item[$\bullet$] Ultrasonic Distance Measuring Sensor Module 
    
    \item[$\bullet$] Breadboard and jumper cables
    
\end{description}

The Mechanical design of the Robot car includes hardware such as motor and wheel placement and body setup. Robot car uses two gear-motors attached to wheels and one free wheel for forward, backward, left and right movements. Free wheel ball is placed at rear side of the robot which helps for 360 degrees free movement. L298N DC Stepper Motor Drive controller is used to control the speed and direction of the two gear motor wheels. Ultrasonic sensors are placed at front side of the robot which is capable to detect the objects on its path.

\section{Data Flow Diagram}
\begin{figure}[ht!]
  \includegraphics[width=0.5\textwidth]{FlowDiagram.png}
  \caption{Data Flow Diagram}
\end{figure}

\section{Technological Review}
Technological Components
\begin{description}
\item[$\bullet$] Raspbian(Operating System)
\item[$\bullet$] Arduino tool
\item[$\bullet$] Wi-Fi
\end{description}

\section{System, Software and Hardware Requirements}
\subsection{Platform Requirement}
\subsubsection{Windows}
\begin{description}
\item[$\bullet$] Window 7 and above Microsoft Windows is a series of
graphical interface operating systems developed, marketed,
and sold by Microsoft.
\item[$\bullet$] Microsoft introduced an operating environment named Windows on November 20, 1985 as a graphical operating system shell for MS-DOS in response to the growing interest in graphical user interfaces(GUIs).
\end{description}


\subsubsection{Android}
\begin{description}
\item[$\bullet$] Android OS Android is a mobile operating system (OS) based on the Linux kernel and currently developed by Google.
\item[$\bullet$] Android is designed primarily for touch screen mobile devices such as smart phones and table computers, with specialized user interfaces.
\end{description}

The Software Requirements in this project include:
1. JDK
2. Arduino
3. Android Studio
Eclipse :
Eclipse is an integrated development
environment (IDE). It contains a base workspace and an
extensible plug-in system for customizing the environment.

\subsubsection{JDK : Java Platform (JDK)}
The Java Development Kit (JDK) is an implementation
of either one of the Java SE, Java EE or Java ME platforms released by Oracle Corporation in the form of a binary product aimed at Java developers on Solaris, Linux, Mac OS X or Windows.
The JDK includes a private JVM and a few other
resources to finish the recipe to a Java Application.

\subsubsection{Arduino}
Arduino is a tool for making computers that can sense and control more of the physical world than your desktop computer. It’s an open-source physical computing platform based on a simple microcontroller board, and a development environment for writing software for the board.

\subsection{HARDWARE REQUIREMENT}
The Hardware components required for our project are Min 1 GB of RAM,10 GB HDD, Dual core processor for the machine on which development will be done, Robot kit, IP Camera, Bluetooth Module etc for developing the robot.

Hardware Requirement
\begin{description}
\item[a)] Android Smartphone
\item[b)] Bluetooth module
\item[c)] Microcontroller(Arduino)
\item[d)] DC power supply
\item[e)] Motor Driver
\item[f)] DC motor
\item[g)] Wires, Breadboard, Small PCB.
\end{description}


\section{Circuit Block Diagram}

\section{Main Component}

\section{Application}
1. Scientific
2. Space Probes
3. Submarines
4. Military and Law Enforcement
5. Recreation and Hobby

\section{Code}
\begin{lstlisting}
/* include library */
#include <ESP8266WiFi.h>
#include <Servo.h> 
#include <NewPing.h>

/* define port */
WiFiClient client;
WiFiServer server(80);

#define SONAR_SERVO_PIN 16
#define TRIGGER_PIN     5
#define ECHO_PIN        4

#define MAX_DISTANCE    200
/* WIFI settings */
const char* ssid = "virusdetected";
const char* password = "HemaMunna424$$";

/* data received from application */
String  data =""; 

/* define L298N or L293D motor control pins */
int rightMotorForward = 13;  /*GPIO15(D8)->IN1*/
int rightMotorBackward = 15; /*GPIO13(D7)->IN2*/
int leftMotorForward = 0;    /*GPIO2(D4)->IN3*/
int leftMotorBackward = 2;   /*GPIO0(D3)->IN4*/



/* define L298N or L293D enable pins */
int rightMotorENB = 14; /*GPIO14(D5)->Motor-A Enable*/
int leftMotorENB = 12;  /*GPIO12(D6)->Motor-B Enable*/

NewPing sonar(TRIGGER_PIN, ECHO_PIN, MAX_DISTANCE);
Servo myServo;

const int triggerDistance = 25;

// Variables
unsigned int time1; // to store how long it takes for the ultrasonic wave to come back
int distance;       // to store the distance calculated from the sensor
int fDistance;  // to store the distance in front of the robot
int lDistance;  // to store the distance on the left side of the robot
int rDistance;  // to store the distance on the right side of the robot


char dist[3];
char rot[3];
int rotation = 0;
String output = "";

void connectWiFi()
{
  Serial.println("Connecting to WIFI");
  WiFi.begin(ssid, password);
  while ((!(WiFi.status() == WL_CONNECTED)))
  {
    delay(300);
    Serial.print("..");
  }
  Serial.println("");
  Serial.println("WiFi connected");
  Serial.println("NodeMCU Local IP is : ");
  Serial.print((WiFi.localIP()));
}

void scan(int deg)
{
  myServo.write(deg);
  delay(10);

  int time1 = sonar.ping();
  distance = time1 / US_ROUNDTRIP_CM;
  delay(10);
  if(distance <= 0){
    distance = triggerDistance;
  }

  delay(30);
} 

/*** FORWARD ***/
void MotorForward(void)   
{
  digitalWrite(leftMotorENB,HIGH);
  digitalWrite(rightMotorENB,HIGH);
  digitalWrite(leftMotorForward,HIGH);
  digitalWrite(rightMotorForward,LOW);
  digitalWrite(leftMotorBackward,LOW);
  digitalWrite(rightMotorBackward,HIGH);
}

/*** BACKWARD ***/
void MotorBackward(void)   
{
  digitalWrite(leftMotorENB,HIGH);
  digitalWrite(rightMotorENB,HIGH);
  digitalWrite(leftMotorBackward,HIGH);
  digitalWrite(rightMotorBackward,LOW);
  digitalWrite(leftMotorForward,LOW);
  digitalWrite(rightMotorForward,HIGH);
}

/*** TURN LEFT ***/
void TurnLeft(void)   
{
  digitalWrite(leftMotorENB,HIGH);
  digitalWrite(rightMotorENB,HIGH); 
  digitalWrite(leftMotorForward,LOW);
  digitalWrite(rightMotorForward,LOW);
  digitalWrite(rightMotorBackward,HIGH);
  digitalWrite(leftMotorBackward,HIGH);  
}

/*** TURN RIGHT***/
void TurnRight(void)   
{
  digitalWrite(leftMotorENB,HIGH);
  digitalWrite(rightMotorENB,HIGH);
  digitalWrite(leftMotorForward,HIGH);
  digitalWrite(rightMotorForward,HIGH);
  digitalWrite(rightMotorBackward,LOW);
  digitalWrite(leftMotorBackward,LOW);
}

/*** STOP ***/
void MotorStop(void)   
{
  digitalWrite(leftMotorENB,LOW);
  digitalWrite(rightMotorENB,LOW);
  digitalWrite(leftMotorForward,LOW);
  digitalWrite(leftMotorBackward,LOW);
  digitalWrite(rightMotorForward,LOW);
  digitalWrite(rightMotorBackward,LOW);
}

/*** RECEIVE DATA FROM the APP ***/
String checkClient (void)
{
  while(!client.available()) delay(1); 
  String request = client.readStringUntil('\r');
  request.remove(0, 5);
  request.remove(request.length()-9,9);
  return request;
}

void setup()
{
  Serial.begin(115200);
  connectWiFi();
  server.begin();

  pinMode(TRIGGER_PIN, OUTPUT);
  pinMode(ECHO_PIN, INPUT);
  myServo.attach(SONAR_SERVO_PIN);  // Attaches the Servo to the Servo Object 
  /* initialize motor control pins as output */
  pinMode(leftMotorForward, OUTPUT);
  pinMode(rightMotorForward, OUTPUT); 
  pinMode(leftMotorBackward, OUTPUT);  
  pinMode(rightMotorBackward, OUTPUT);

  /* initialize motor enable pins as output */
  pinMode(leftMotorENB, OUTPUT); 
  pinMode(rightMotorENB, OUTPUT);

  /* start server communication */
}

void loop()
{
  
  scan(90);                                //Get the distance retrieved
  fDistance = distance;
  if(fDistance < triggerDistance){
    MotorBackward();
    delay(682); 
    MotorStop();
    scan(170);
    delay(600);
    lDistance = distance;
    scan(3);
    delay(600);
    rDistance = distance;
   if(lDistance < rDistance){
      TurnRight();
      delay(682);
      MotorStop();
      MotorForward();
    }
    else{
      TurnLeft();
      delay(682);
      MotorStop();
      MotorForward();
    }
  }
  else{
    MotorForward();
  } 
}




\end{lstlisting}

\section{Goals and Objectives}
''Robot Controlled Car Using Wireless Module'' at the title
suggests is aimed to construct a control system that enables the
complete control of the interface of Machine.
It include two separate units :-
\begin{description}
\item[$\bullet$] The Android Phone

\item[$\bullet$] Control Unit.
\end{description}
Two Operating Environments :-
\begin{description}
\item[$\bullet$] The android phone will operate indoors

\item[$\bullet$] Outdoors whereas the control unit will operate
indoors within the temperature and humidity limits for proper
operation of the hardware.
\end{description}




\section{Conclusion}

\begin{acks}

The authors would like to thank Dr. Gregor von Laszewski for his support and suggestions in writing this paper.

\end{acks}

%\bibliographystyle{ACM-Reference-Format}
%\bibliography{report} 

\end{document}
