\documentclass[sigconf]{acmart}

\input{format/final}

\graphicspath{ {images/} }

\usepackage{listings}

\lstset{aboveskip=3mm,
  belowskip=3mm,
  showstringspaces=false,
  columns=flexible,
  basicstyle={\small\ttfamily},
  breaklines=true,
  breakatwhitespace=true,
  tabsize=3
}

\begin{document}

\title{Face Detection and Recognition Using Raspberry Pi Robot Car}

\author{Mani Kumar Kagita}
\affiliation{%
  \institution{Indiana University}
  \streetaddress{107 S. Indiana Avenue}
  \city{Bloomington} 
  \state{Indiana} 
  \postcode{43017-6221}
}
\email{mkagita@iu.edu}


\begin{abstract}
Face recognition is an exciting and emerging field of computer vision with many  applications to hardware and devices. Using embedded platforms like the Raspberry Pi, a camera module and open source computer vision libraries like OpenCV, purpose is to add face recognition to Robot car and also facial recognition using free developer version of Kairos Facial Recognition software.
In today`s modern world, face recognition playing an important role for the purpose of security and surveillance and hence there is a need for an efficient and cost effective system. So the main goal is to explore the feasibility of implementing Raspberry Pi based facial recognition system using conventional face detection and recognition techniques such as Haarcascade detection and Kairos. An obstacle avoidance Robot car is integrated with Raspberry Pi and a camera module aiming at taking face recognition to a level in which the system can identify the humans who are stuck in buildings during earth quakes.  Raspberry Pi kit provides the system cost effective and easy to use, with high performance.

\end{abstract}

\keywords{Raspberry Pi, Robot Car, Face Recognition, Face Identification, I523, HID319}

\maketitle

\section{Introduction}
A Computer Vision application which has always encouraged people, concern about the capability and capacity of  robots and computers to determine, detect, recognize and interact with human beings \cite{Boris2014}. We will prevail the advantage of cheaper tools that are available in market for computing and detecting human face from the image, recognizing the face using hardware like Raspberry Pi and a video camera that is dedicated to Raspberry Pi. Simple and open source software like OpenCV is used to detect human face from the video that is being captured and the image will be sent to Kairos facial recognition software which allows a high level approach to this process.

In this fastest information era, every information is travelled in split of seconds. There is much more need for accurate and fastest methods in identifying, recognizing and authentication of humans. In the present world, Facial recognition had became most important and crucial form of human identification methods. As per Literature survey statistics in face recognition, the two trends to receive significant attention for the past several years are; the first is the law enforcement applications and also wide range of commercial techniques, and the second is exponential booming of applications and feasible technologies after 30 years of research \cite{riddhi2013}.

The aim is achieved by a possibility to locate human beings or their parts like faces from the live video capture and within the pictures context. Most advanced human detection applications have this functionality already available. When the picture is capture and loaded into the system, it will scan the picture and will look for human faces in it. Current implementation is to detect face and register them with a name. If the face is detected and not recognized, Robot car will ask to register the detected face with a name. If the human is already registered in Kairos, then once the face is detected, Robot car will greet the human with the associated name.This whole process determines the Face detection and Face recognition techniques using Raspberry Pi and Robot car.

Facial biometric data is to be computed first in creating a complete recognition system. This biometric data is then compared with the face database and to associate with the human identity. The difference between a human and machine is, a human can easily and quickly identify characteristics of a human face but then can only save few hundreds of faces. Whereas a machine or computers prevails at storing and mapping human characteristics and meta data. In current generation, facial recognition softwares can identify a human face with in millions of images from the database in seconds. Humans tend to forget human faces as time pass by. Machines stores them forever. Most of the Law firms across the world follow the process and spend huge money with development of these facial recognition systems that can easily identify criminals in real-time. A well-known example is studying human faces in airports and bus stations.

The design of the Robot car integrated with Face recognition system will navigate through dangerous or natural disaster locations where humans unable to enter. Robot car while avoiding obstacles on its way, will continuously monitor for human faces who got stuck or in danger and will recognize the faces based on the user database. Once the human face is recognized, it will intimate to corresponding authorities about the human and will help in guiding assistance. 

\section{Face Detection}
Face Detection is a technique referred to computer vision technology which is able to identify human faces within digital images \cite{divya2013}. Face detection applications works using algorithms and machine learning formulas for detecting human faces in the visual images. Identifying only human faces from these images which can contains landscapes, houses, animals is called Face Detection technique.

Face Detection is termed to only identify if there are any humans present in the image or a video. It lacks in ability to recognize which human face is present. Common widely used face detection techniques are in auto-focus of a digital camera. During auto-focus, camera lens will look for human faces in the range and identify them to have focus in that particular area.
Face Detection techniques will be widely used in counting how many number of visitors attending a particular event.

\subsection{How Face Detection Works}
While Face Detection process is somewhat complex, the algorithms will start off by searching for human eyes at first. Eyes usually represents a valley region and its the easiest feature in human face to detect. Once the eyes are detected, then the algorithms will look for rest of the characteristics of a human face such as iris, nose, mouth, eyebrows and nostrils. Face detection algorithm then summarizes the data and shows that it has successfully detected a human face from the facial region. An additional tests can be conducted by the algorithm to make sure and validate if its detected human face or not \cite{jesse2017}.

\section{Face Recognition}
Like most of the biometrics solutions, face recognition technology will be used for identification and authentication purposes by measuring and matching the unique facial characteristics of a human face. Using a digital camera connected to raspberri pi, once the face is detected, facial recognition software will quantify the characteristics of face and then will match with the stored images in database. Once the match is positive, then the corresponding name will be displayed as output \cite{biometrics2016}.

Face biometrics can be integrated to any system having a camera. Border control agencies use face recognition to verify identities of the travellers and can seperate them from the trespassers. Government Law agencies replace all the security cameras around the world with biometric applications to scan faces in CCTV footage, and to identify persons of interest in the field. Face recognition has become one of the fastest and human unintervention techniques to find out the identity of a particula human \cite{biometrics2016}.

For the past few years, Face recognition has became on of the most commonly used bio-metric authentication techniques. It maily deals with the Pattern recognition and analyzing the images. Two main tasks of Facial recognition are: Face verification and Face Identification. Face Verification is comparing a human face in an image with a template image and recognizing the correct patterns. Face Identification is comparing human face in an image with multiple images in the database. Face recognition techniques have more advantages than any other biometrics. With well sophesticated algorithms and coading, Face recognition has a high recognition rate or high identification rate of more than 90\% \cite{riddhi2013}. 

\begin{figure}[ht!]
  \includegraphics[width=\columnwidth]{images/Face-recognition.jpg}
  \caption{Block Diagram of a Face Recognition System}
\end{figure}


\section{Software and Hardware Specifications}
OpenCV is to be installed in Raspberry Pi to detect human faces with in the captured images. Kairos Facial recognition software is used to recognize human face and identify with the corresponding name.

\subsection{Software Used}

\subsubsection{Raspian OS}
This is the recommended OS for Raspberry Pi 3.  Raspbian OS is debian based OS. It can be installed from noobs installer. Raspbian comes with pre-installed softwares such as Python, Sonic Pi, Java, Mathematica for programming and education.
\subsubsection{Putty}
PuTTY is an SSH and telnet client, developed originally by Simon Tatham for the Windows platform. PuTTY is open source software that is available with source code and is developed and supported by a group of volunteers. Here we are using putty for accessing our raspberry pi remotely.

\subsubsection{OpenCV}
OpenCV (Open Source Computer Vision Library) is an open source computer vision and machine learning software library. OpenCV was built to provide a common infrastructure for computer vision applications and to accelerate the use of machine perception in the commercial products. Being a BSD-licensed product, OpenCV makes it easy for businesses to utilize and modify the code. The library has more than 2500 optimized algorithms, which includes a comprehensive set of both classic and state-of-the-art computer vision and machine learning algorithms. These algorithms can be used to detect and recognize faces, identify objects, classify human actions in videos, track camera movements, track moving objects and extract 3D models of objects \cite{opencv}.

\subsubsection{Python 2 IDE}
Python 2.7.x version Integrated Development Environment is used to compile python program in Raspberry Pi. IDE is a text editor plus terminal combination which is used to work on large projects with complex code bases.

\subsubsection{Kairos Facial Recognition Software}
Kairos is an artificial intelligence company specializing in face recognition. Through computer vision and machine learning, Kairos can recognize faces in videos, photos, and the real-world. A captured image is sent to Kairos using an API call and then Kairos will search with the face database. If it matches then will reply with the human name.
\begin{description}
  \item[$\bullet$] Identity 
  \item[$\bullet$] Emotions 
  \item[$\bullet$] Demographics
\end{description}

Kairos navigates the complexities of face analysis technology.

\subsection{Hardware Used}
\subsubsection{Raspberry Pi 3}
Raspberry Pi 3 is the latest version of Raspberry Pi. Unless previous versions, this have an unbuilt Bluetooth platform and a wi-fi support module. There are total 40 pins in RPI3. Of the 40 pins, 26 are GPIO pins and the others are power or ground pins (plus two ID EEPROM pins.) There are 4 USB Port and 1 Ethernet slot, one HDMI port, 1 audio output port and 1 micro usb port and also many other things you can see the diagram on right side. And also we have one micro sd card slot wherein we have to installed the recommended Operating system on micro sd card. There are two ways to interact with your raspberry pi. Either you can interact directly through HDMI port by connecting HDMI to VGA cable, and keyboard and mouse or else you can interact from any system through SSH(Secure Shell) \cite{deligence2017} 

\subsubsection{Raspberry Pi Camera}
The Raspberry Pi camera module can be used to take high-definition video, as well as stills photographs. It`s easy to use for beginners, but has plenty to offer advanced users if you’re looking to expand your knowledge. There are lots of examples online of people using it for time-lapse, slow-motion and other video cleverness. You can also use the libraries we bundle with the camera to create effects.

\subsubsection{Robot Car Chassis Kit}
The Mechanical design of the Robot car includes hardware such as motor and wheel placement and body setup. Robot car uses two gear-motors attached to wheels and one free wheel for forward, backward, left and right movements. Free wheel ball is placed at rear side of the robot which helps for 360 degrees free movement \cite{arduino2015}. L298N DC Stepper Motor Drive controller is used to control the speed and direction of the two gear motor wheels. Ultrasonic sensors are placed at front side of the robot which is capable to detect the objects on its path.



\section{System Architecture}
System Architecture consists of following blocks :
\begin{description}
\item[a)] Raspberry Pi
\item[b)] Raspberry Pi Camera Module
\item[c)] L298N Dual H-Bridge Stepper Motor Controller
\item[d)] DC power supply 12v and 5v
\item[e)] Robot Car chassis kit
\item[f)] HC-SR04 Ultrasonic Sensor
\item[g)] SG90 Servo Motor.
\item[h)] Wires, Breadboard, Small PCB.
\end{description}

The Mechanical design of the Robot car includes hardware such as motor and wheel placement and body setup. Robot car uses two gear-motors attached to wheels and one free wheel for forward, backward, left and right movements. Free wheel ball is placed at rear side of the robot which helps for 360 degrees free movement. L298N DC Stepper Motor Drive controller is used to control the speed and direction of the two gear motor wheels. Ultrasonic sensors are placed at front side of the robot which is capable to detect the objects on its path. Raspberry Pi Camera module is used to monitor the live stream and recognize the face if its detected.

\section{Setup}
\subsection{Connect Raspberry Pi}
This section includes connectivity of Raspberry Pi over Wifi. 
\begin{description}

    \item[$\bullet$] Download Raspbian OS to an SD card with a minimum capacity of 8GB.
    
    \item[$\bullet$] Plug in USB power cable, keyboard, mouse and monitor cables to Raspberry Pi.
    
    \item[$\bullet$] Insert the SD card with Raspbian OS into Pi and boot the system. Once the Pi is booted up, a window will appear with Raspbian operating system. Click on Raspbian and Install.
    
    \item[$\bullet$] When the install process has completed, the Raspberry Pi configuration menu (raspi-config) will load. Here  set the time and date for your region.
    
    \item[$\bullet$] Enable wifi on upper right corner and connect to wifi sid.
\end{description}

\subsection{Connect Raspberry Pi Camera Module}
\begin{description}
    \item[$\bullet$] Install the Raspberry Pi Camera module by inserting the cable into the Raspberry Pi.
    \item[$\bullet$] The cable slots into the connector situated between the Ethernet and HDMI ports, with the silver connectors facing the HDMI port.
    \item[$\bullet$] Boot up your Raspberry Pi and run below commands in command prompt.
    \item[$\bullet$] sudo apt-get install python-pip
    \item[$\bullet$] sudo apt-get install python-dev
    \item[$\bullet$] sudo pip install picamera
    \item[$\bullet$] sudo pip install rpio
    \item[$\bullet$] From the prompt, run ''sudo raspi-config''. 
    \item[$\bullet$] If the ''camera'' option is not listed, you will need to run a few commands to update your Raspberry Pi. Run ''sudo apt-get update'' and ''sudo apt-get upgrade''
\end{description}
\subsubsection{Enable Camera}
For Face Detaction, PiCamera should be enable from Raspberry Pi. Below list of figures shows the detailed steps on how to enable PiCamera from Raspberry Pi.

\begin{figure}[htb]
  \includegraphics[width=\columnwidth]{images/enablecamera1.jpg}
  \caption{Edit raspi-config file from command line}
%\end{figure}

\bigskip

%\begin{figure}[ht!]
  \includegraphics[width=\columnwidth]{images/enablecamera2.jpg}
  \caption{Select Camera from the options}
%\end{figure}

\bigskip

%\begin{figure}[ht!]
  \includegraphics[width=\columnwidth]{images/enablecamera3.jpg}
  \caption{Enable Camera}
\end{figure}

\subsection{Install OpenCV and Required Libraries}
OpenCV computer vision library is used to perform face detection and recognition. For this, first need to install OpenCV dependencies on Raspberry Pi. Below commands needs to be executed. 
\begin{description}
    \item[$\bullet$] sudo apt-get update
    \item[$\bullet$] sudo apt-get upgrade
    \item[$\bullet$] sudo apt-get install build-essential
    \item[$\bullet$] cmake pkg-config python-dev libgtk2.0-dev libgtk2.0 zlib1g-dev libpng-dev libjpeg-dev libtiff-dev libjasper-dev libavcodec-dev swig unzip
    \item[$\bullet$] Select yes for all options and wait for the libraries and dependencies to be installed
\end{description}
Download opencv-2.4.9 zip file to Raspberry Pi. Change the directory and execute cmake command as follows:

\begin{verbatim}
  cd opencv-2.4.9
  sudo apt-get install build-essential cmake pkg-config
  sudo apt-get install libjpeg-dev libtiff5-dev libjasper-dev libpng12-dev
  sudo apt-get install python-dev python-numpy libtbb2 libtbb-dev \    
                       libjpeg-dev libpng-dev libtiff-dev \
                       libjasper-dev libdc1394-22-dev
  sudo apt-get install python-opencv
  sudo apt-get install python-matplotlib

\end{verbatim}

After executing the commands the latest version of OpenCV is now installed in Raspberry Pi.

\subsection{Integration of Raspberry Pi with Robot Car}
Raspberry Pi connected with PiCamera is integrated with Robot car to navigate using webserver. During the navigation, robot car will look for human faces using PiCamera and then detects the face. Once the face is detected, python program will call Kairos facial detection software to identify the person and greet with the name. If the human face is unidentified then robot car will ask human to register their name.

As shown in the figure below, connect a Robot car chassis to raspberry pi and follow the circuit connections.

\begin{figure}[ht!]
  \includegraphics[width=\columnwidth]{images/RaspPi_Robot.jpg}
  \caption{Raspberry Pi Robot Car Integration}
\end{figure}

In Table \ref{T:pinlayout} we show ....

\begin{table}[htb]
\caption{WHat is this}\label{T:pinlayout}
\begin{tabular}{lll}
Actuator & Pin & ???? \\
\hline
    Motor1A & 16 & GPIO 23 - Pin 16 \\
    Motor1B & 18 & GPIO 24 - Pin 18 \\
    Motor1Enable & 22 & GPIO 25 - Pin 22 \\
    Motor2A & 21 & GPIO 9 - Pin 21 \\
    Motor2B & 19 & GPIO 10 - Pin 19 \\
    Motor2Enable & 23 & GPIO 11 - Pin 23 \\
\end{tabular}
\end{table}

\subsection{Kairos Face Recognition Setup}
Kairos Face Recognition system has a free developer account which is used to identify the human name from the images. Once registered a human name with an image, the code will call Kairos API with a newly detected human face and will look for the registered name. Kairos will do a quick lookup in human database from the registered account and if it matches, will send the name of the human back to the code.

Setup as follows:

\begin{description}
    \item[$\bullet$] Register will https:/www.kairos.com as a free developer account
    \item[$\bullet$] Login with regisetered username and password
    \item[$\bullet$] Create an Appname
    \item[$\bullet$] An App ID and a key are generated. Save this for future use
    \item[$\bullet$] Enroll users and a gallery name with the user images
    \begin{lstlisting}
    POST /enroll HTTP/1.1
    Content-Type: application/json
    app_id: your-app-id
    app_key: your-app-key
    {
    "image":" http://media.kairos.com/user.jpg ",
    "subject_id":"User",
    "gallery_name":"MyGallery"
    }
    \end{lstlisting}
\end{description}

\section{Code Explanation}
\subsection{Face Detection}
Import all the related libraries including PiCamera and PiRGBArray libraries for camera to operate in Raspberry Pi. These libraries will help to capture video and images from the PiCamera.

\begin{lstlisting}
from picamera.array import PiRGBArray
from picamera import PiCamera
import time
import cv2
import sys
import imutils
from fractions import Fraction
import base64
import requests
import json
import random
import os
\end{lstlisting}

Haarcascade is a tool to capture the frontal features of face. This tool will help to continuous monitoring for any human face to detect. Once detected a human face, the output values will provide as Human Face Detected from the capturing video.

\begin{lstlisting}
# Get user supplied values
cascPath = './haarcascade_frontalface_default.xml'

# Create the haar cascade
faceCascade = cv2.CascadeClassifier(cascPath)
\end{lstlisting}

Camera settings needs to be updated in the code as per below suggestions. The capture image is to be sent to Kairos for Facial recognition and so we will set the resolution to a lower level. This will help to send the image faster over the network without any delay.

\begin{lstlisting}
# initialize the camera and grab a reference to the raw camera capture
camera = PiCamera()
camera.resolution = (160, 120)
camera.framerate = 32
rawCapture = PiRGBArray(camera, size=(160, 120))
\end{lstlisting}

Below code represents PiCamera continously monitor for human faces detected from the grayscale video capture. Once the human face is detected, espeak function in Raspberry Pi will send the voice to a connected speaker and will output as ``Human face detected``. This detected image is then saved as ``User-Image.jpg`` which is then will be sent to Kairos during Face recognition.

Here are the front, sideviews of the face detected images.

\begin{figure}[ht!]
  \includegraphics[width=\columnwidth]{images/Face-detect-frontview.png}
  \caption{Front View of Face detection}
\end{figure}

\begin{figure}[ht!]
  \includegraphics[width=\columnwidth]{images/Face-detect-sideview1.png}
  \caption{Side view 1 of Face detection}
\end{figure}

\begin{figure}[ht!]
  \includegraphics[width=\columnwidth]{images/Face-detect-sideview2.png}
  \caption{Side view 2 of Face detection}
\end{figure}


\begin{lstlisting}
# allow the camera to warmup
time.sleep(0.1)
lastTime = time.time()*1000.0
# capture frames from the camera
for frame in camera.capture_continuous(rawCapture, format="bgr", use_video_port=True):
	# grab the raw NumPy array representing the image, then initialize the timestamp
	# and occupied/unoccupied text
    image = frame.array
    gray = cv2.cvtColor(image, cv2.COLOR_BGR2GRAY)
    
    # Detect faces in the image
    faces = faceCascade.detectMultiScale(
    gray,
    scaleFactor=1.1,
    minNeighbors=5,
    minSize=(30, 30),
    flags = cv2.cv.CV_HAAR_SCALE_IMAGE
    )
    print time.time()*1000.0-lastTime," Found {0} faces!".format(len(faces))
    lastTime = time.time()*1000.0

    # Draw a circle around the faces
    for (x, y, w, h) in faces:
        cv2.circle(image, (x+w/2, y+h/2), int((w+h)/3), (255, 255, 255), 1)
    # show the frame
    cv2.imshow("Frame", image)
    key = cv2.waitKey(1) & 0xFF
    if len(faces) == 1:
        print("Taking image...")
	camera.capture("foo.jpg")
	os.system('espeak "Human face detected"')
	inputImage= "./foo.jpg"
	del camera
	break 
	# clear the stream in preparation for the next frame
    rawCapture.truncate(0)
    
	# if the `q` key was pressed, break from the loop
    if key == ord("q"):
        del camera
        exit()
\end{lstlisting}


\subsection{Face Recognition}
For the Face Recognition, we use Kairos to detect the facial characteristics. A json config file is to be placed in the same folder as of the code with Kairos API app id and key value. When the human face is detected, code will generate an API call to Kairos software along with the gallery name, API app id and key values. Image when sending to Kairos, it will be base64 encrypted and will send over the network for security purpose. This encrypted image will then be decrypted at Kairos platform.


\begin{lstlisting}
KAIROS = "api.kairos"
KairosGallery = 'MyFace'
KairosConfig = './kairos_config.json'
\end{lstlisting}

\begin{lstlisting}
def trainKairos(image, name):
    global KairosGallery
    headers = {
        'app_id': 'your-app-idd39fc1b1',
        'app_key': 'your-app-key'
    }
    data = {
        'image': base64.b64encode(image),
        'gallery_name': KairosGallery,
        'subject_id': name
    }
    r = requests.post('http://api.kairos.com/enroll', headers=headers, data=json.dumps(data))
    print(r.text)
    return(None)
\end{lstlisting}

\begin{lstlisting}
class Recognize():
    def __init__(self, API, config_file):
        self.api = API
        self.config = config_file

    #def recognize(self, image_path):
    #    return self.__recognizeKairos(image_path)
    
    def recognizeKairos(self, image):
        with open(image, "rb") as image_file:
            encoded_string = base64.b64encode(image_file.read())
        with open(self.config, "rb") as config_file:
            config = json.loads(config_file.read())
        data = {
            "image": encoded_string,
            "gallery_name": config["gallery_name"]
        }

        headers = {
            "Content-Type": "application/json",
            "app_id": config["app_id"],
            "app_key": config["app_key"]
        }
\end{lstlisting}

Output from Kairos software is in json format. The output is then segregated as per the key value pairs and then saved into local variables. When the image is recognized, a success transaction message will be obtained from Kairos along with subject id and face id.

\begin{lstlisting}
try:
    r = requests.post("https://api.kairos.com/recognize", headers=headers, data=json.dumps(data))
    data = r.json()
    print data
    # print json.dumps(data, indent=4)
    faces = []
    if "images" in data:
        for obj in data["images"]:
            if obj["transaction"]["status"] == "success":
                face_obj = {}
                face_obj["person"] = obj["transaction"]["subject_id"]
                .decode("utf_8")
                #face_obj["faceid"] = obj["candidates"][0]["face_id"]
                .decode("utf_8")
                face_obj["confidence"] = obj["transaction"]["confidence"]
                faces.append(face_obj)
            elif obj["transaction"]["status"] == "failure":
                face_obj = {}
                face_obj["person"] = "unidentified"
                face_obj["confidence"] = 0
                faces.append(face_obj)
            else:
                print "its in last loop"
            return faces
 except requests.exceptions.RequestException as exception:
       print exception
        return None
\end{lstlisting}    

Output from Kairos face recogniion software is to be read to understand if the person name is identified or not. If its identified then the person name sill be listed according to the corresponding person in the image. If the human is not identified, then code will suggest if the user wants to registered for face recognition. Once the user key in the name, Kairos API call is generated while sending newly registered name and the gallery name to that corresponding app id. Here the newly recognized user will be registerd with the name and his image. When the user is reconized by camera in next corresponding events, then Robot car will greet the user with his name.

\begin{lstlisting}
if __name__ == "__main__":
    r = Recognize(KAIROS, "kairos_config.json")
    x = r.recognizeKairos(inputImage)
    
    #print x
    #print x["person"]
    #print x[0]["person"]
    string1 = x[0]["person"]
    #print string1
    os.system('espeak "Hello...""{}"'.format(string1))
    if x[0]["person"] == "unidentified":
        os.system('espeak "Please enter your name to Register"')
        nameToRegister = raw_input("Please enter your name to Register :")
        binaryData = open(inputImage, 'rb').read()
        print('Enrolling to Kairos')
        trainKairos(binaryData, nameToRegister)
        print "You are now Registered as :", nameToRegister
        os.system('espeak "Hello...""{}"'.format(nameToRegister))
        exit()
\end{lstlisting}

\subsection{Robot Car Navigation}
\begin{lstlisting}
import RPi.GPIO as GPIO
from time import sleep

GPIO.setmode(GPIO.BOARD)
\end{lstlisting}

\begin{lstlisting}
#Connecting two wheel motors to Raspberry Pi GPIO 
#Left Motor (Motor 1) connections
Motor1A = 16 #(GPIO 23 - Pin 16)
Motor1B = 18 #(GPIO 24 - Pin 18)
Motor1Enable = 22 #(GPIO 25 - Pin 22)

#Right Motor (Motor 2) Connecctions
Motor2A = 21 #(GPIO 9 - Pin 21)
Motor2B = 19 #(GPIO 10 - Pin 19)
Motor2Enable = 23 #(GPIO 11 - Pin 23)
\end{lstlisting}

\begin{lstlisting}
#Ouptut of Morors to set as OUT
GPIO.setup(Motor1A,GPIO.OUT)
GPIO.setup(Motor1B,GPIO.OUT)
GPIO.setup(Motor1Enable,GPIO.OUT)
GPIO.setup(Motor2A,GPIO.OUT)
GPIO.setup(Motor2B,GPIO.OUT)
GPIO.setup(Motor2Enable,GPIO.OUT)

\end{lstlisting}

\begin{lstlisting}
# Defining function for Robot car to move forward
def forward():
	GPIO.output(Motor1A,GPIO.HIGH)
	GPIO.output(Motor1B,GPIO.LOW)
	GPIO.output(Motor1Enable,GPIO.HIGH) 
	GPIO.output(Motor2A,GPIO.HIGH)
	GPIO.output(Motor2B,GPIO.LOW)
	GPIO.output(Motor2Enable,GPIO.HIGH) 

	sleep(2)
\end{lstlisting}

\begin{lstlisting}
# Defining function for Robot car to move backward
def backward():
	GPIO.output(Motor1A,GPIO.LOW)
	GPIO.output(Motor1B,GPIO.HIGH)
	GPIO.output(Motor1Enable,GPIO.HIGH)
	GPIO.output(Motor2A,GPIO.LOW)
	GPIO.output(Motor2B,GPIO.HIGH)
	GPIO.output(Motor2Enable,GPIO.HIGH)

	sleep(2)
\end{lstlisting}

\begin{lstlisting}
# Defining function for Robot car to turn right
def turnRight():
	print("Going Right")
	GPIO.output(Motor1A,GPIO.HIGH)
	GPIO.output(Motor1B,GPIO.LOW)
	GPIO.output(Motor1Enable,GPIO.HIGH)
	GPIO.output(Motor2A,GPIO.LOW)
	GPIO.output(Motor2B,GPIO.LOW)
	GPIO.output(Motor2Enable,GPIO.LOW)

	sleep(2)
\end{lstlisting}

\begin{lstlisting}
# Defining function for Robot car to turn left
def turnLeft():
	print("Going Left")
	GPIO.output(Motor1A,GPIO.LOW)
	GPIO.output(Motor1B,GPIO.LOW)
	GPIO.output(Motor1Enable,GPIO.LOW)
	GPIO.output(Motor2A,GPIO.HIGH)
	GPIO.output(Motor2B,GPIO.LOW)
	GPIO.output(Motor2Enable,GPIO.HIGH)

	sleep(2)
\end{lstlisting}

\begin{lstlisting}
# Defining function for Robot car to stop
def stop():
	print("Stopping")
	GPIO.output(Motor1A,GPIO.LOW)
	GPIO.output(Motor1B,GPIO.LOW)
	GPIO.output(Motor1Enable,GPIO.LOW)
	GPIO.output(Motor2A,GPIO.LOW)
	GPIO.output(Motor2B,GPIO.LOW)
	GPIO.output(Motor2Enable,GPIO.LOW)
\end{lstlisting}

\subsection{Controling Robot Car using webserver}
\begin{lstlisting}
from flask import Flask, render_template, request, redirect, url_for, make_response
import RPi.GPIO as GPIO
import motors

#set up GPIO
GPIO.setmode(GPIO.BOARD) 

#set up flask server
app = Flask(__name__) 

#when the root IP is selected, return index.html page
@app.route('/')
def index():

	return render_template('index.html')
\end{lstlisting}

\begin{lstlisting}
#recieve which pin to change from the button press on index.html
#each button returns a number that triggers a command in this function
#
#Uses methods from motors.py to send commands to the GPIO to operate the motors
@app.route('/<changepin>', methods=['POST'])
def reroute(changepin):

	changePin = int(changepin) #cast changepin to an int

	if changePin == 1:
		motors.turnLeft()
	elif changePin == 2:
		motors.forward()
	elif changePin == 3:
		motors.turnRight()
	elif changePin == 4:
		motors.backward()
	else:
		motors.stop()


	response = make_response(redirect(url_for('index')))
	return(response)

#set up the server in debug mode to the port 8000
app.run(debug=True, host='0.0.0.0', port=8000) 
\end{lstlisting}

\section{Applications}
There are lots of applications of face recognition. Face recognition is already being used to unlock phones and specific applications. Face recognition is also used for biometric surveillance. Banks, retail stores, stadiums, airports and other facilities use facial recognition to reduce crime and prevent violence.

\section{Conclusion}
A Face detection and a recognition system is developed using Raspberry Pi. Using Python programming language, system is being built such that it can face detect and recognize in real time scenarios. For this solution Kairos Facial recognition software is being used which have a free developer account. Facial recognition is tested with various types of faces ie, front view, sideview. The Round Trip Time for robot car to take picture and recognize face is nearly 3seconds. Efficiency of the system was alalyzed based on the rate of face detection in real time. As per this alalysis, this current system shows tremendous performance efficiency where the face detection and recognition can be performed even with a very low quality images.


\begin{acks}

The authors would like to thank Dr. Gregor von Laszewski for his support and suggestions in writing this paper.

\end{acks}

\bibliographystyle{ACM-Reference-Format}
\bibliography{report} 

\end{document}
